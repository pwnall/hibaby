\section{Related Work}
\label{sec:related}

Online privacy and anonymity has been a hotbed of academic research.
More specifically, dating sites have been shown to be an exciting application with a new cost function and emphasis on user privacy and security~\cite{okws}.


\section{De-Anonymization Of Users of Social Networks}
\label{sec:related_anon}

De-anonymization (also referred to as re-identification) of users on a social network is a high-value research area, and has recieved much attention in recent years, and has been treated increasingly as its own branch of computer science~\cite{reidentification1}.
An growing understanding that a secure, private web is somewhat of an illusion~\cite{webisnotsecure} prompted many users to pay attention to the privacy policy and security measures of the web applications they use~\cite{motivationforanonymity}.
A large body of research, including but not limited to~\cite{deanonsocial1,deanonsocial2,locationprivacy} has been conducted in de-anonymizing users on the internet in order to demystify the illusion of privacy, and better inform future security work.
While profile photos have been identified as a risk to privacy and anonymity~\cite{profilephotos1}, we are not aware of any successful attack on anonymity of a social network via analysis of profile pictures.

Web crawling as a means of data harvesting for large-scale (``big-data'') analysis has been popular with market research, advertising, sentiment analysis, and other applied fields~\cite{crawling1}.
Web application operators seldom derive benefit from web crawling, and may occasionally lose business due to competitors' data gathering, and understandably invest in detecting and curbing large-scale data collection~\cite{limitcrawling1, limitcrawling2, protectidentity1, protectidentity2} and non-human users.

\section{Identifying identical images}
\label{sec:related_images}


Quick identification of similar images is an open research area with many applications ranging from medical imaging to search.~\cite{nearduplicateimage1,nearduplicateimage2}. In this work, we rely on the pHash~\cite{phash} library to implement image comparison, but pHash is unable to align differently cropped images for a robust comparison. There exists a rich body of work for image registration, including ready-to-use solutions~\cite{registration3}.
