\section{Design}
\label{sec:design}

In this section, we outline our method for de-anonymizing DMS by aggregating profiles with Facebook data, using profile photos to identify accounts.

\subsection{Impostor Accounts}
\label{sec:design_accounts}

We begin by constructing a fictitious identity and create the accounts necessary to gather information from DMS and Facebook.
The name and photograph are a simple matter of choosing a common name and a profile photograph (using an average-of-faces face)[TODO: ref big bang theory]. Let \textbf{FU} (``faux-user'') denote the new identity consisting of a fictitious profile photo and a name, as well as some arbitrary profile information, and a made-up birth date.

In order to create a DMS account, \textbf{FU} needs a school-affiliated email (we generated a new \texttt{@mit.edu} address), and a Facebook account - a simple matter of manually filling out a web form and clicking a verification link emailed by Facebook. We are now able to create a fictitious DMS profile for \textbf{FU} by, again, simply filling out a web form.
We do not use existing accounts to avoid compromising our identities, and to avoid unfair advantage such as using a established social network to obtain additional user information.

\subsection{Impersonating Human Users}
\label{sec:design_debug}

We are now able to access both DMS and Facebook using \textbf{FU}'s fictitious account.
Obviously, to de-anonymize DMS profiles on the large scale, we must be able to automate all repeated interaction with the site.
Similarly, we must be able to programmatically extract a large set of Facebook profiles expected to overlap with the extracted DMS profiles.
Creating HTTP sessions within a Python or Ruby script is easy, but is not robust against detection, throttling, and is therefore good approach to extracting large amounts of information from either DMS or Facebook.
While many sophisticated techniques exist to instrument or impersonate human user sessions, ranging from screen manipulation \cite{yeh2009sikuli} to browser impersonation over HTTP requests (TODO: cite), we found our approach to be highly effective and very simple.
We use the Chromium remote debugging interface~\cite{chrome-rdb} to instrument a real browser session and programmatically specify user input.
This approach to web crawling uses each site's built-in search functionality in the same way a human user would, meaning the web crawler sessions are virtually impossible to detect at the server.
We effectively impersonate a tablet user, thereby obviating the need to falsify mouse hover data.

The web crawler is coded in as generic a way as possible, as we expect it to be widely applicable to other crawlable bodies of information online.
For example, a small wrapper over the same the same code base is used to log into both Facebook and DMS, as given in Figure~\ref{fig:login-code}.

\begin{figure*}
\begin{lstlisting}[language=Ruby]
def dms_login(client, email, password)
  client.page_events = true
  client.navigate_to 'https://datemyschool.com'
  client.wait_for type: WebkitRemote::Event::PageLoaded

  login_link = client.dom_root.query_selector '#login-link'
  click_element client, login_link

  login_forms = client.dom_root.query_selector_all 'form[action*="login"]'
  if login_forms.length != 1
    raise 'Login form selector broken or not specific enough'
  end
  login_form = login_forms.first

  email_input = login_form.query_selector 'input[type="text"]'
  input_text client, email_input, email

  password_input = login_form.query_selector 'input[type="password"]'
  input_text client, password_input, password

  submit_button = login_form.query_selector 'input[type="submit"]'
  click_element client, submit_button

  client.wait_for type: WebkitRemote::Event::PageLoaded
  client.page_events = false
  client.clear_all
end

def fb_login(client, email, password)
  client.page_events = true
  client.navigate_to 'https://facebook.com'
  client.wait_for type: WebkitRemote::Event::PageLoaded

  login_forms = client.dom_root.query_selector_all 'form[action*="login"]'
  if login_forms.length != 1
    raise 'Login form selector broken or not specific enough'
  end
  login_form = login_forms.first

  email_input = login_form.query_selector 'input[type="text"]'
  input_text client, email_input, email

  password_input = login_form.query_selector 'input[type="password"]'
  input_text client, password_input, password

  submit_button = login_form.query_selector 'input[type="submit"]'
  click_element client, submit_button

  client.wait_for type: WebkitRemote::Event::PageLoaded
  client.page_events = false
  client.clear_all
end
\end{lstlisting}
\caption{Crawler login code for DMS and Facebook.}
\label{fig:login-code}
\end{figure*}

In crawling DMS, we use the crawler to automatically login, navigate to the search page, and enumerate URLs for all profiles affiliated with a given school. We also download and extract all profile photos affiliated with each account, and store an entry consisting of \texttt{[profile link, photo(s)]} for each DMS user at a given school in an unordered set.

Crawling Facebook is slightly more challenging, as it does not quite allow all members of a school to be enumerated easily. We are, however, able to enumerate all users in the group affiliated with the school - a crude but effective approximation.
Just as with DMS, we extract all profile photographs associated with each account we record.
We store an entry consisting of \texttt{[name, email, photo(s)]} for each Facebook account expected to overlap with the DMS data set we created above.

\subsection{Identifying Profiles}
\label{sec:design_profile}

We now have two (large) data sets of user profiles: anonymous DMS profiles consisting of $D = $\texttt{[profile link, photo(s)]}, and non-anonymous Facebook profiles consisting of $F = $\texttt{[name, email, photo(s)]}.
We expect the ``de-anonymized'' set $U = D \cap F$  (the set of users of both sites) to be significant in size.
To find profiles in $d_u \in D$ and $f_u \in F$ such that $d_u$ and $f_u$ both belong to the same user $u$, we must find a means of correlating anonymous profiles in $D$ with non-anonymous profiles in $F$.

Prior work has relied on text analysis to classify profile authors in order to de-anonymize social networks, but dating sites unfortunately seldom provide a large writing sample.
Unable to rely on text, we instead use other information to correlate profiles.
Fortunately, we found a deceptively simple and accurate means of matching profiles across $D$ and $F$: the profile photos.

We theorize that users rely on a small set of ``good'' photographs for profile photos where they attempt to look their best.
When creating a DMS profile, the users must find a recent high-quality photograph, and we conjecture rely on their Facebook profile photos to select a recent ``good'' image.
As demonstrated in our Results~\ref{sec:results}, the conjecture holds for our initial experiments.
We expect the profile pictures to have been resized and recoded in a lossy way from the original photographs uploaded by the users.
While we cannot simply compare the files, we are nevertheless able to efficiently compare profile photos using the pHash~\cite{phash} library, calculating perceptual hashes of each photo in $D$ and $F$.
Finally, we simply enumerate all pairs $(d_u \in D, f_u \in F)$ such that $d_u$ and $f_u$ have a ``similar'' photo hash (meaning with overwhelming probability an identical profile photo).
