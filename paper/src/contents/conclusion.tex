\section{Conclusion}
\label{sec:conclusion}

Sites operating under the promise of anonymity, be it explicit or implied, should do more to educate their users.
Users place some trust in a site when creating a profile in that they delegate the responsibility of maintaining privacy to the site.
These users often fail to understand that the information they share publicly may be enough to de-anonymize their profile, making any mesures the site implements to protect their privacy irrelevant.

By aggregating 
If an attacker is able to identify users 


 that the site works to maintain their promise of privacy, but often fail to understand that the information they make publically 


In this work, we demonstrate a simple technique to defeat the anonymizing measures of DMS by impersonating a legitimate user using a Chromium remote debugging interface and extracting user data for an entire locale in bulk.
We then use straightforward crawling of a non-anonymous social network (in this work, we use Facebook) to extract public photos and basic identifying information in the same locale.
Conjecturing that many users re-use a small set of photographs for profile pictures, we use the pHash library to 

A few simple 
A simple extension to the flow for creating an anonymous profile would protect 

(DMS already requests a user's Facebook account a sign-up, meaning the site has all information needed to warn vulnerable users)

is easy).


Sites that promise anonymity should do more to educate their users. Check pic
against facebook pictures using phash. (dms asks for fb during signup anyway)
Check that profile name doesn't include real name.

\section{Future Work}
\label{sec:related_futurework}

Bla